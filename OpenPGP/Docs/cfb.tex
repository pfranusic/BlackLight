%@.tex
%%%% cfb.src
%%%% OpenPGP Cipher Feedback (CFB) mode
%%%% Copyright 2012 Peter Franusic
%%%%

\documentclass{article}
%\pagestyle{empty}

%%%% Various environments
\usepackage{verbatim}
\usepackage{graphicx}
\usepackage{latexsym}
\usepackage{amsmath}
\usepackage{amssymb}

%%%% Easy-vision mode
%\usepackage[usenames]{color}
%\pagecolor{black}
%\color{green}

%%%% math-mode commands
\newcommand{\xbyte}{\mathrm{byte}}
\newcommand{\xqeq}{\overset{?}{=}}

%%%% European-style paragraphs
%%%% IMPORTANT: \begin{document} must follow for this to work.
\setlength{\parindent}{0pt} 
\setlength{\parskip}{1.3ex} 
\begin{document}


%%%%%%%%%%%%%%%%%%%%%%%%%%%%%%%%%%%%%%%%%%%%%%%%%%%%%%%%%%%%%%%%
\textbf{\huge{CFB}}
\vspace{8ex}

\emph{Cipher feedback} (CFB) is a block cipher mode of operation.

A \emph{block cipher} encrypts plaintext in short blocks.
For example, AES-128 is a block cipher that encrypts plaintext in 128-bit blocks.
A plaintext message is split into plaintext blocks $P_i$,
a block cipher $\kappa$ encrypts each plaintext block into a ciphertext block $C_i$,
and each ciphertext block is then transmitted.

Plaintext messages often contain well-known strings; an example is an email header.
Well-known strings can be a big problem, even if they are encrypted.
For instance, if a block cipher is used without something like CFB, 
a cryptanalyst can collect ciphertexts of email headers, compile a codebook,
and then spoof encrypted emails, all without knowing the key.

CFB mode uses the same encryption function in both CFB encode and CFB decode functions.
For example, if an CFB encode function calls the AES-128 encryption function,
then the inverse CFB decode function also calls the AES-128 encryption function.
Decryption functions are not used in CFB mode.

\section*{Literature}

CFB mode was originally used in conjunction with the Data Encryption Standard (DES) 
and is specified in the 1980 document \emph{DES Modes of Operation} along with tables of examples.
[FIPS PUB 81].

Schneier discusses CFB mode in his 1994 book \emph{Applied Cryptography} in section 8.1.4,
and Figure 8.5 offers two block diagrams for 8-bit CFB mode, one for encryption and 
the other for decryption. [Schneier 1994]

Menezes et al discuss CFB mode in their 1996 \emph{Handbook of Applied Cryptography}
in section 7.2.2 (iii).  Figure 7.1 (c) offers two block diagrams for r-bit CFB mode,
and Algorithm 7.17 provides the details. [HAC 1999]

CFB mode is also specified in the 2001 NIST document \emph{Recommendation for Block Cipher 
Modes of Operation} in section 6.3 and illustrated in Figure 3.  The document also 
provides example vectors for CFB-128 encipherment and AES-128 encryption in Appendix F.3.13
and CFB-128 decipherment and AES-128 encryption in Appendix F.3.14. [NIST SP 800-38A]


%%%%%%%%%%%%%%%%%%%%%%%%%%%%%%%%%%%%%%%%%%%%%%%%%%%%%%%%%%%%%%%%
\section*{Electronic codebook}

To develop an understanding of CFB mode, 
it's best to start with the simplest block cipher mode.
This is \emph{electronic codebook} mode (ECB).
The ECB encoder computes each ciphertext block $C_i$ by simply
applying the block cipher function $\kappa$ to each plaintext block $P_i$.
\[ C_i = \kappa(P_i) \]

The ECB decoder recovers each plaintext block $P_i$ by
applying the inverse block cipher function $\hat{\kappa}$ 
to each ciphertext block $C_i$.
\[ P_i = \hat{\kappa}(C_i) \]

Schneier identifies several shortfalls of ECB.
``The problem with ECB mode is that if cryptanalysts have the plaintext and ciphertext
for several messages, they can start to compile a codebook without knowing the key.
In most real-world situations, fragments of messages tend to repeat \ldots
This vulnerability is greatest at the beginning and end of messages,
where well-defined headers and footers contain information about the sender,
receiver, date, etc.''

Another problem is the block replay vulnerability.
Mallet opens two bank accounts, one with Bank 1 and the other with Bank 2.
He sets up a recording device on the EFT channel between the two banks.
He orders funds transfers from his Bank 1 account to his Bank 2 account
and identifies the corresponding ECB blocks on the channel.
Then he simply replays those blocks on the channel and watches his Bank 2 account grow.

These vulnerabilities do not exist in cipher feedback mode,
which uses a simple function to augment the block cipher.


%%%%%%%%%%%%%%%%%%%%%%%%%%%%%%%%%%%%%%%%%%%%%%%%%%%%%%%%%%%%%%%%
\section*{Cipher feedback}

The cipher feedback mode uses the exclusive-OR operator
and a feedback loop to introduce complexity.
The exclusive-OR operator is represented by the symbol $\oplus$.
It is also abbreviated \emph{xor} (pronounced ``ex or'').
It has several useful features: the identity element is 0,
the inverse of any element $A$ is itself, and 
the associative property holds.
\begin{eqnarray}
  A \oplus 0 &=& A \\
  A \oplus A &=& 0 \\
  (A \oplus B) \oplus C &=& A \oplus (B \oplus C)
\end{eqnarray}

The CFB encode function computes the current ciphertext block $C_i$
by applying the encryption function $\kappa$ 
to the previous ciphertext block $C_{i-1}$
and then xor-ing that with the current plaintext block $P_i$.
\[ C_i = P_i \oplus \kappa(C_{i-1}) \]

The CFB decode function recovers the current plaintext block $P_i$
by applying the encryption function $\kappa$ 
to the previous ciphertext block $C_{i-1}$
and then xor-ing that with the current ciphertext block $C_i$.
Note that the encryption function $\kappa$ is used
rather than the decryption function $\hat{\kappa}$.
\[ P_i = C_i \oplus \kappa(C_{i-1}) \]

The equation for $P_i$ can be derived from $C_i$.
We start with the equation for $C_i$ and ``add'' $\kappa(C_{i-1})$ to both sides.
The associative property lets us shift the square brackets on the right side.
The two xored $\kappa(C_{i-1})$ terms cancel because $A \oplus A = 0$.
And we can drop the 0 because $A \oplus 0 = A$.
Rearranging, we get the equation for $P_i$.
\begin{eqnarray*}
  C_i  &=&  P_i \oplus \kappa(C_{i-1}) \\
  C_i \oplus \kappa(C_{i-1})  &=&
      \big[ P_i \oplus \kappa(C_{i-1}) \big] \; \oplus \; \kappa(C_{i-1}) \\
  C_i \oplus \kappa(C_{i-1})  &=&
      P_i \; \oplus \; \big[ \kappa(C_{i-1}) \oplus \kappa(C_{i-1}) \big] \\
  C_i \oplus \kappa(C_{i-1})  &=&  P_i \oplus 0 \\
  C_i \oplus \kappa(C_{i-1})  &=&  P_i \\
  P_i  &=&  C_i \oplus \kappa(C_{i-1}) \qquad \Box
\end{eqnarray*}

The CFB encode function computes successive ciphertext blocks $C_i$.
It takes a plaintext block $P_i$ and xors it with 
an encryption of the previous ciphertext block, $\kappa(C_{i-1})$.
This works great except for the initial case where we compute $C_0$.
That is, what do we use for $C_{-1}$?  It doesn't exist!
\[ C_0 = P_0 \oplus \kappa(C_{-1}) \]

The CFB decode function computes successive plaintext blocks $P_i$.
It takes a ciphertext block $C_i$ and xors it with 
an encryption of the previous ciphertext block, $\kappa(C_{i-1})$.
We have the same problem with $C_{-1}$.
\[ P_0 = C_0 \oplus \kappa(C_{-1}) \]

And we have another problem with CFB decode.
It's possible to decode thousands of blocks of ciphertext with the wrong key.
We need some way to verify, at the beginning of CFB decode,
that we have the correct key.

The $C_{-1}$ problem is solved by replacing $\kappa(C_{-1})$ with $\kappa(0)$.
The key verification problem is solved by using a random block $R$.
With these two solutions, we can write complete algorithms 
for the basic CFB encode and decode functions.


%%%%%%%%%%%%%%%%%%%%%%%%%%%%%%%%%%%%%%%%%%%%%%%%%%%%%%%%%%%%%%%%
\section*{Basic CFB} 

We need a random block $R$.  For a block size of n bits, 
$R$ is a randomly-selected value anywhere between 0 and $2^n-1$ inclusive.
\[ R \in \{ 0, 1, \ldots, (2^n-1) \} \]

\begin{quote}
\textbf{Basic CFB encoder:}\\
Select a random $R$.
Compute and output both preliminary blocks $C_x$ and $C_y$.
Then input plaintext block $P_i$ and output ciphertext block $C_i$.
\begin{eqnarray*}
  R    &=&  \mathrm{rand}(2^n) \\
  C_x  &=&  R  \oplus  \kappa(0)  \\
  C_y  &=&  R  \oplus  \kappa(C_x)  \\
  C_0  &=&  P_0  \oplus  \kappa(C_y)  \\
  C_1  &=&  P_1  \oplus  \kappa(C_0)  \\
  C_2  &=&  P_2  \oplus  \kappa(C_1)  \\
  &\vdots&
\end{eqnarray*}

\textbf{Basic CFB decoder:}\\
Input ciphertext blocks $C_x$ and $C_y$.
Compute both $R$ and $R'$ and make sure they're equal.
Then input ciphertext block $C_i$ and output plaintext block $P_i$.
\begin{eqnarray*}
  R  &=&  C_x  \oplus  \kappa(0)  \\
  R' &=&  C_y  \oplus  \kappa(C_x)  \\
  R  &\xqeq&  R' \\
  P_0  &=&  C_0  \oplus  \kappa(C_y)  \\
  P_1  &=&  C_1  \oplus  \kappa(C_0)  \\
  P_2  &=&  C_2  \oplus  \kappa(C_1)  \\
  &\vdots&
\end{eqnarray*}
\end{quote}

In OpenPGP, these two algorithms are more complicated.
They require three preliminary blocks instead of two.
$C_x$ is the same, but $C_y$ is truncated, 
and an additional \emph{resynchronization} block $C_z$ is required.


%%%%%%%%%%%%%%%%%%%%%%%%%%%%%%%%%%%%%%%%%%%%%%%%%%%%%%%%%%%%%%%%
\section*{OpenPGP CFB}

The definitive guide for OpenPGP CFB mode is RFC-4880.
Section 13.9 provides the following twelve step algorithm for the CFB encode function.
An algorithm for the CFB decode function is not specified in RFC-4880.
Unfortunately, the authors describe the algorithm without mathematical notation.
Instead they employ a sort of register transfer language.
FR is the feedback register. FRE is the encrypted feedback register.
IV is the initialization vector.
They also use the term ``octet'' instead of ``byte'' which is more common.
BS is the block size, either 8 octets (64 bits) or 16 octets (128 bits).
\begin{quote}
\begin{enumerate}

\item The feedback register (FR) is set to the IV, which is all zeros.

\item FR is encrypted to produce FRE (FR Encrypted).
      This is the encryption of an all-zero value.

\item FRE is xored with the first BS octets of random data prefixed to the plaintext
      to produce C[1] through C[BS], the first BS octets of ciphertext.

\item FR is loaded with C[1] through C[BS].

\item FR is encrypted to produce FRE,
      the encryption of the first BS octets of ciphertext.

\item The left two octets of FRE get xored with the next two octets of
      data that were prefixed to the plaintext.  This produces C[BS+1]
      and C[BS+2], the next two octets of ciphertext.

\item (The resynchronization step) FR is loaded with C[3] through C[BS+2].

\item FR is encrypted to produce FRE.

\item FRE is xored with the first BS octets of the given plaintext, now
      that we have finished encrypting the BS+2 octets of prefixed data.
      This produces C[BS+3] through C[BS+(BS+2)], the next BS octets of ciphertext.

\item FR is loaded with C[BS+3] to C[BS + (BS+2)] (which is C11-C18 for an 8-octet block).

\item FR is encrypted to produce FRE.

\item FRE is xored with the next BS octets of plaintext, to produce
      the next BS octets of ciphertext.  These are loaded into FR, and
      the process is repeated until the plaintext is used up.

\end{enumerate}
\end{quote}


%%%%%%%%%%%%%%%%%%%%%%%%%%%%%%%%%%%%%%%%%%%%%%%%%%%%%%%%%%%%%%%%
\section*{Distillation}

The twelve-step procedure may be distilled.
Consider that the crucial steps are 3, 6, 9, and 12.
These are the steps where output is produced.
Step 3 produces ``C[1] through C[BS].''
Step 6 produces ``C[BS+1] and C[BS+2].''
Step 9 produces ``C[BS+3] through C[BS+(BS+2)].''
Step 12 produces ``the next BS octets of ciphertext.''

We specify a distilled procedure with BS = 16 and AES-128 encryption.
The distilled procedure uses only four steps (A, B, C, D).
Step D is repeated until the plaintext is exhausted.
The Plaintext register (P) contains all of the plaintext octets (bytes).
The AES session key register (K) contains a 16-byte (128-bit) value.
The Ciphertext register (C) is initially empty.
The Feedback Register (FR) and the Feedback Register Encrypted (FRE) are 16-byte values.
The Random value register (R) contains a 16-byte random value.
The most significant byte of a register has index 1.
For example, the most significant byte of FRE is FRE[1].

\begin{quote}
\textbf{Step A: C[1] C[2] ... C[16]}\\
The FR is reset to all zeros.
Then the contents of FR is AES-128 encrypted using the session key K
and the 16-byte result is written into FRE.
The 16 bytes in FRE is xored with the 16 bytes in R
and the 16-byte result is written to C[1] through C[16].
\vspace{2ex}

\textbf{Step B: C[17] C[18]}\\
The FR is set to the 16-byte value in C[1] through C[16].
Then the contents of FR is AES-128 encrypted using the session key K
and the 16-byte result is written into FRE.
The two bytes in FRE[1] and FRE[2] are xored with the two bytes in R[15] and R[16]
and the two-byte result is written to C[17] and C[18].
\vspace{2ex}

\textbf{Step C: C[19] C[20] ... C[34]}\\
The FR is set to the 16-byte value in C[3] through C[18].
Then the contents of FR is AES-128 encrypted using the session key K
and the 16-byte result is written into FRE.
The 16 bytes of FRE is xored with the 16 bytes of P[1] through P[16]
and the 16-byte result is written to C[19] through C[34].
\vspace{2ex}

\textbf{Step D: C[35] C[36] ... C[50]}\\
The FR is set to the value in C[19] through C[34].
Then the contents of FR is AES-128 encrypted using the session key K
and the 16-byte result is written into FRE.
The 16 bytes of FRE is xored with the 16 bytes of P[17] through P[32]
and the 16-byte result is written to C[35] through C[50].
\end{quote}

The algorithm in RFC-4880 does not specify what to do when 
the number of plaintext bytes is not a multiple of the block size BS.
But it turns out that Steps C and D must be slightly modified to account for these cases.
If the number of bytes remaining in the plaintext queue is $0 < n < 16$,
then the 16 bytes of FRE are xored with a special 16-byte padded block,
but only $n$ ciphertext octets are actually emitted.
The 16-byte padded block consists of the remaining $n$ bytes and $16-n$ padding bytes.
The positions of these bytes does not matter as long as the decoder follows that same pattern.


%%%%%%%%%%%%%%%%%%%%%%%%%%%%%%%%%%%%%%%%%%%%%%%%%%%%%%%%%%%%%%%%
\section*{Encode algorithm}

Now we specify the algorithm for the OpenPGP CFB encode function.
We use the same four steps as above.
But instead of register transfer language, we use mathematical notation.
This avoids the feedback register and the encrypted feedback register.
We use uppercase variables for 128-bit values (e.g. $C_0$)
and lowercase variables for 8-bit values (e.g. $c_{19}$).

\begin{quote}
\textbf{Step A:}\\
Select a random value for $R$ between 0 and $2^{128}$.
Compute $C_x$ as the xor of $R$ and $\kappa(0)$.
Split $C_x$ into 16 ciphertext bytes and output them as $c_{1}\ c_{2}\ \ldots\ c_{16}$.
\begin{eqnarray*}
  R &=& \mathrm{rand}(2^{128}) \\
  C_x &=& R \oplus \kappa(0) \\
  c_j &=& \xbyte_{16-j}(C_x) \qquad j \in [1,16] \\
\end{eqnarray*}

\textbf{Step B:}\\
Compute $C_y$ as the xor of the most-significant word (msw) 
of the encryption of $C_x$ with the least-significant word (lsw) in $R$.
Split $C_y$ into 2 ciphertext bytes and output them as $c_{17}\ c_{18}$.
\begin{eqnarray*}
  C_y &=& \mathrm{msw}(\kappa(C_x)) \oplus \mathrm{lsw}(R) \\
  c_{17} &=& \xbyte_{1}(C_y) \\
  c_{18} &=& \xbyte_{0}(C_y) \\
\end{eqnarray*}

\textbf{Step C:}\\
Initialize $i$ to 0.
Compute $C_z$ as the lower 7 words of $C_x$ multiplied by 65536 and then added to $C_y$.
Let $n$ be the number of remaining plaintext bytes.
Input up to 16 plaintext bytes $p_{1}\ p_{2}\ \ldots$ and merge them into $P_0$.
If $n<16$ then pad $P_0$ in the least-significant bytes.
Compute $C_0$ as the xor of $P_0$ and $\kappa(C_z)$.
Split $C_0$ into 16 ciphertext bytes and output $n$ of them as $c_{19}\ c_{20}\ \ldots$
\begin{eqnarray*}
  i &=& 0 \\
  C_z &=& 65536 \cdot \mathrm{L}_7(C_x) + C_y  \\
  P_0 &=& \sum_{j=1}^{16} 256^{16-j} \cdot p_{j} \\
  C_0 &=& P_0 \oplus \kappa(C_z) \\
  c_{18+j} &=& \xbyte_{16-j}(C_0) \qquad j \in [1,16] \\
\end{eqnarray*}

\textbf{Step D:}\\
Increment $i$.
Let $n$ be the number of remaining plaintext bytes.
If $n=0$ then terminate.
Input up to 16 plaintext bytes $p_{16i+1}\ p_{16i+2}\ \ldots$ and merge them into $P_i$.
If $n<16$ then pad $P_i$ in the least-significant bytes.
Compute $C_i$ as the xor of $P_i$ and $\kappa(C_{i-1})$.
Split $C_i$ into 16 ciphertext bytes and output $n$ of them as $c_{16i+19}\ c_{16i+20}\ \ldots$
Repeat this step.
\begin{eqnarray*}
  i &=& i + 1 \\
  P_i &=& \sum_{j=1}^{16} 256^{16-j} \cdot p_{16i+j} \\
  C_i &=& P_i \oplus \kappa(C_{i-1}) \\
  c_{16i+18+j} &=& \xbyte_{16-j}(C_i) \qquad j \in [1,16] \\
\end{eqnarray*}
\end{quote}

%%%%%%%%%%%%%%%%%%%%%%%%%%%%%%%%%%%%%%%%%%%%%%%%%%%%%%%%%%%%%%%%
\section*{Decode algorithm}

As mentioned above, RFC-4880 does not specify an algorithm for the OpenPGP CFB decode function.
However, it is easy enough to derive the decoder equations from the encoder equations.
For $R$, $P_0$, and $P_i$ we use the same trick that we used in the proof, 
where we add $\kappa(X)$ to both sides.
For $C_y$ and $C_z$ we use the same equation as the encoder function.

\begin{quote}
\textbf{Step A:}\\
Input 16 ciphertext bytes $c_{1}\ c_{2}\ \ldots\ c_{16}$ and merge them into $C_x$.
Then compute the random block $R$ as the xor of $C_x$ and $\kappa(0)$.
\begin{eqnarray*}
  C_x &=& \sum_{j=1}^{16} 256^{16-j} \cdot c_{j} \\
  R &=& C_x \oplus \kappa(0) \\
\end{eqnarray*}

\textbf{Step B:}\\
Compute $C_y$ as the xor of the most-significant word (msw) 
of the encryption of $C_x$ with the least-significant word (lsw) in $R$.
Input the 2 ciphertext bytes $c_{17}\ c_{18}$ and merge them into $C_y'$.
Verify that $C_y$ and $C_y'$ are equal.
\begin{eqnarray*}
  C_y  &=& \mathrm{msw}(\kappa(C_x)) \oplus \mathrm{lsw}(R) \\
  C_y' &=& 256 \cdot c_{17} + c_{18} \\
  C_y  &\xqeq& C_y' \\
\end{eqnarray*}

\textbf{Step C:}\\
Initialize $i$ to 0.
Compute $C_z$ as the lower 7 words of $C_x$ multiplied by 65536 and then added to $C_y$.
Let $n$ be the number of remaining ciphertext bytes.
Input up to 16 ciphertext bytes $c_{19}\ p_{20}\ \ldots$ and merge them into $C_0$.
If $n<16$ then pad $C_0$ in the least-significant bytes.
Compute $P_0$ as the xor of $C_0$ and $\kappa(C_z)$.
Split $P_0$ into 16 plaintext bytes and output $n$ of them as $p_{1}\ p_{2}\ \ldots$
\begin{eqnarray*}
  i &=& 0 \\
  C_z &=& 65536 \cdot \mathrm{L}_7(C_x) + C_y  \\
  C_0 &=& \sum_{j=1}^{16} 256^{16-j} \cdot c_{18+j} \\
  P_0 &=& C_0 \oplus \kappa(C_z) \\
  p_j &=& \xbyte_{16-j}(P_0) \qquad j \in [1,16] \\
\end{eqnarray*}

\textbf{Step D:}\\
Increment $i$.
Let $n$ be the number of remaining ciphertext bytes.
If $n=0$ then terminate.
Input up to 16 ciphertext bytes $c_{16i+19}\ c_{16i+20}\ \ldots$ and merge them into $C_i$.
If $n<16$ then pad $C_i$ in the least-significant bytes.
Compute $P_i$ as the xor of $C_i$ and $\kappa(C_{i-1})$.
Split $P_i$ into 16 plaintext bytes and output $n$ of them as $p_{16i+1}\ p_{16i+2}\ \ldots$
Repeat this step.
\begin{eqnarray*}
  i &=& i + 1 \\
  C_i &=& \sum_{j=1}^{16} 256^{16-j} \cdot c_{16i+18+j} \\
  P_i &=& C_i \oplus \kappa(C_{i-1}) \\
  p_{16i+j} &=& \xbyte_{16-j}(P_i) \qquad j \in [1,16] \\
\end{eqnarray*}
\end{quote}



\end{document}
%%%%%%%%%%%%%%%%%%%%%%%%%%%%%%%%%%%%%%%%%%%%%%%%%%%%%%%%%%%%%%%%




%@.tex
%%%%%%%%%%%%%%%%%%%%%%%%%%%%%%%%%%%%%%%%%%%%%%%%%%%%%%%%%%%%%%%%
\section*{\texttt{cfb-128-encode}}

Input is a 128-bit session key integer k and a list of plaintext bytes p-list.
Output is a list of ciphertext bytes c-list.
Expands the session key k into a 44-word key schedule,
performs the AES-128 Cipher function in OpenPGP CFB Mode on all bytes in p-list,
and returns c-list.

%@.cl
\begin{verbatim}
(defun cfb-128-encode (k p-list)
  (let ((w) (n) (r) (c-list))
    
    ;; Make sure the inputs are okay.
    ;; Create key schedule w from session key k.
    (if (not (and (integerp k) (plusp k) (< (log k 2) 128)))
	(error "k must be a positive integer less than 2^128"))
    (if (not (blistp p-list))
	(error "p-list must be a non-empty list of bytes"))
    (setf w (aes-128-expand k))

    ;; Step A:
    ;; Select a random value for r.
    ;; Compute cx as the xor of r and k(0).
    ;; Split cx into 16 bytes and output them.
    (setf r (random (expt 2 128)))
    (setf cx (logxor (aes-128-cipher w 0) r))
    (setf c-list (split-int 16 cx))

    ;; Step B:
    ;; Compute cw as the encryption of cx.
    ;; Compute cy as the xor of the msw and the lsw.
    ;; Split cy into 2 bytes and append them to c-list.
    (setf cw (aes-128-cipher w cx))
    (setf cy (logxor (quo cw (expt 2 112)) (rem r (expt 2 16))))
    (setf c-list (append c-list (split-int 2 cy)))

    ;; Step C:
    ;; Compute cz = 65536*L7(cx) + cy.
    ;; Let n be the number of remaining plaintext bytes.
    ;; Input up to 16 plaintext bytes and merge them into p.
    ;; If n < 16 then pad p in the least-significant bytes.
    ;; Compute c0 as the xor of p0 and k(cz).
    ;; Output n bytes of c0.
    (setf cz (unite-int (subseq c-list 2 18)))
    (setf n (length p-list))
    (setf p0 0)
    (dotimes (i n)
      (setf p0 (+ (* 256 p0) (pop p-list))))
    (dotimes (i (- 16 n))
      (setf p0 (* 256 p0)))
    (setf c0 (logxor p0 (aes-128-cipher w cz)))
    (setf c-list (append c-list (subseq (split-int 16 c0) 0 n)))

    ;; Step D:
    ;;; While p-list is not empty:
    ;;; Let n be the number of remaining bytes in p-list.
    ;;; Compute the 128-bit integer $p_i$ from the remaining n bytes,
    ;;; trailed by (- 16 n) dummy bytes in the least-significant bits.
    ;;; Compute the 128-bit integer $c_i = aes(c_{i-1}) + p_i$.
    ;;; Split $c_i$ into 16 bytes and append the first n bytes to c-list.
    ;;; Remove n bytes from p-list.
    (while (> (setf n (length p-list)) 0)
      (setf f (logxor (aes-cipher-128 w f)
		      (unite-int (append p-list (listn (- 16 n) 0)))))
      (setf c-list (append c-list (subseq (split-int 16 f) 0 n)))
      (dotimes (i n) (pop p-list)))
    ;;; Return c-list.
    c-list))
\end{verbatim}


\begin{comment}
%%%%%%%%%%%%%%%
Input is a 128-bit session key integer k and a list of plaintext bytes p-list.
Output is a list of ciphertext bytes c-list.

  Step A:
  Make sure the inputs are okay.
  Create key schedule w from session key k.
  Create r-list with 16 random byte values.

  Step B:
  Compute the 128-bit integer $r_i$ from r-list.
  Compute the 128-bit integer $c_{-2} = aes(0) + r_i$
  Split $c_{-2}$ into 16 bytes and initialize c-list.

  Step C:
  Compute the 16-bit integer $r_q$ from (subseq r-list 14 16).
  Compute the 16-bit integer $c_q = msw(aes(c_{-2})) + r_q$.
  Split $c_q$ into 2 bytes and append them to c-list.

  Step D:
  Compute the 128-bit integer $c_{-1}$ from (subseq c 2 18).

  Step E:
  While there are at least 16 bytes in p-list:
  Compute the 128-bit integer $p_i$ from the next 16 bytes in p-list.
  Compute the 128-bit integer $c_i = aes(c_{i-1}) + p_i$
  Split $c_i$ into 16 bytes and append them to c-list.
  Remove 16 bytes from p-list.

  Step F:
  While p-list is not empty:
  Let n be the number of remaining bytes in p-list.
  Compute the 128-bit integer $p_i$ from the remaining n bytes,
  trailed by (- 16 n) dummy bytes in the least-significant bits.
  Compute the 128-bit integer $c_i = aes(c_{i-1}) + p_i$.
  Split $c_i$ into 16 bytes and append the first n bytes to c-list.
  Remove n bytes from p-list.

  Step G:
  Return c-list.

%%%%%%%%%%%%%%%
\end{comment}


%@.tex
%%%%%%%%%%%%%%%%%%%%%%%%%%%%%%%%%%%%%%%%%%%%%%%%%%%%%%%%%%%%%%%%
\section*{\texttt{cfb-128-decode}}
Input is a 128-bit session key integer k and a list of ciphertext bytes c-list.
Output is a list of plaintext bytes p-list.
Expands the session key k into a 44-word key schedule,
performs the AES-128 Cipher function in OpenPGP CFB Mode
on all bytes in c-list, and returns p-list.

%@.cl
\begin{verbatim}
(defun cfb-128-decode (k c-list)
  (let ((w) (f) (n) (c_i) (p_i) (p-list))
    ;;; Make sure the inputs are okay.
    ;;; Create key schedule w from session key k.
    (if (not (and (integerp k) (plusp k) (< (log k 2) 128)))
	(error "k must be a positive integer less than 2^128"))
    (if (not (and (blistp c-list) (>= (length c-list) 18)))
	(error "c-list must be a list of bytes with at least 18 elements"))
    (setf w (aes-expand-128 k))
    ;;; Construct the 128-bit integer $c_{-1}$ from (subseq c 2 18).
    ;;; Remove 18 bytes from c-list.
    (setf c_i (unite-int (subseq c-list 2 18)))
    (dotimes (i 18) (pop c-list))
    ;;; While there are at least 16 bytes in c-list:
    ;;; Compute the 128-bit integer $c_i$ from the next 16 bytes in c-list.
    ;;; Compute the 128-bit integer $p_i = aes(c_{i-1}) + c_i$.
    ;;; Split $p_i$ into 16 bytes and append them to p-list.
    ;;; Remove 16 bytes from c-list.
    (while (>= (length c-list) 16)
      (setf f c_i) ; Set feedback register to previous $c_i$.
      (setf c_i (unite-int (subseq c-list 0 16)))
      (setf p_i (logxor (aes-cipher-128 w f) c_i))
      (setf p-list (append p-list (split-int 16 p_i)))
      (dotimes (i 16) (pop c-list)))
    ;;; While c-list is not empty:
    ;;; Let n be the number of remaining bytes in c-list.
    ;;; Compute the 128-bit integer $c_i$ from the remaining n bytes,
    ;;; trailed by (- 16 n) dummy bytes in the least-significant bits.
    ;;; Compute the 128-bit integer $p_i = aes(c_{i-1}) + c_i$.
    ;;; Split $p_i$ into 16 bytes and append the first n bytes to p-list.
    ;;; Remove n bytes from c-list.
    (while (> (setf n (length c-list)) 0)
      (setf f c_i)
      (setf c_i (unite-int (append (subseq c-list 0 n) (listn (- 16 n) 0))))
      (setf p_i (logxor (aes-cipher-128 w f) c_i))
      (setf p-list (append p-list (subseq (split-int 16 p_i) 0 n)))
      (dotimes (i 16) (pop c-list)))
    ;;; Return p-list.
    p-list))
\end{verbatim}


\begin{comment}
%%%%%%%%%%%%%%%
Input is a 128-bit session key integer k and a list of ciphertext bytes c-list.
Output is a list of plaintext bytes p-list.

  Step A:
  Make sure the inputs are okay.
  Create key schedule w from session key k.

  Step D:
  Compute the 128-bit integer $c_{-1}$ from (subseq c 2 18).
  Remove 18 bytes from c-list.

  Step E:
  While there are at least 16 bytes in c-list:
  Compute the 128-bit integer $c_i$ from the next 16 bytes in c-list.
  Compute the 128-bit integer $p_i = aes(c_{i-1}) + c_i$.
  Split $p_i$ into 16 bytes and append them to p-list.
  Remove 16 bytes from c-list.

  Step F:
  While c-list is not empty:
  Let n be the number of remaining bytes in c-list.
  Compute the 128-bit integer $c_i$ from the remaining n bytes,
  trailed by (- 16 n) dummy bytes in the least-significant bits.
  Compute the 128-bit integer $p_i = aes(c_{i-1}) + c_i$.
  Split $p_i$ into 16 bytes and append the first n bytes to p-list.
  Remove n bytes from c-list.

  Step G:
  Return p-list.

%%%%%%%%%%%%%%%
\end{comment}


%@.tex
%%%%%%%%%%%%%%%%%%%%%%%%%%%%%%%%%%%%%%%%%%%%%%%%%%%%%%%%%%%%%%%%
\section*{\texttt{cfb-okayp}}
cfb-okayp tests the cfb-128-encode and cfb-128-decode functions.
It selects the random 128-bit key k, builds a list x of plaintext bytes x,
computes a list y of encoded bytes using key k and plaintext x, and
computes a list z of decoded bytes using key k and ciphertext y.
cfb-okayp returns true iff lists x and z are equal.

%@.cl
\begin{verbatim}
(defun cfb-okayp ()
  (let ((k) (x) (y) (z))
    (setf k (random (expt 2 128)))
    (setf x (random-bytes (+ 100 (random 16))))
    (setf y (cfb-128-encode k x))
    (setf z (cfb-128-decode k y))
    (equal x z)))
\end{verbatim}


\section*{Equivalent version}

The following is an equivalent version of the OpenPGP CFB algorithm.
It is written using Lisp syntax and specifies the algorithm for AES-128,
where the plaintext is processed in 128-bit chunks (16 bytes).

First the data components:\\
Session key k is a 128-bit integer.
Plaintext p is a list of 8-bit integers, the plaintext bytes.
Key schedule w is a vector of forty-four 32-bit integers derived from session key k.
Random values r is a list of eighteen 8-bit integers, where the first sixteen bytes 
(nth 0 r) through (nth 15 r) are random values, and the last two bytes (nth 16 r) 
and (nth 17 r) are copies of bytes (nth 14 r) and (nth 15 r).
Feedback register f is a 128-bit integer.
Encrypted register e is a 128-bit integer.
Ciphertext c is a list of 8-bit integers, the ciphertext bytes. It's initially empty.

Next the two AES functions we'll need:\\
aes-expand-128 takes session key k and computes and returns key schedule w.
aes-cipher-128 takes key schedule w and a 128-bit plaintext integer,
and returns a 128-bit ciphertext integer.
Key schedule w must exist before aes-cipher-128 can be called.
The idea is that aes-expand-128 is called once,
after which aes-cipher-128 is called a bunch of times.

The OpenPGP CFB algorithm requires seven initialization steps.
The integer 0 is encrypted using aes-cipher-128 (steps 1 and 2 from above).
The 128-bit product is xored with a 128-bit integer 
composed of the first 16 random bytes in r,
and the 128-bit sum is split into 16 bytes and appended to ciphertext list c (step 3).
The feedback register f is loaded with the same 128-bit sum (step 4).
Now things get slightly convoluted.

The 128-bit value in f in encrypted using aes-cipher-128
and the 128-bit product is written into e (step 5).
The 16 most-significant bits of e are xored with a 16-bit integer
composed of the last two bytes in r, and the 16-bit product is 
split into 2 bytes and appended to ciphertext list c (step 6).
Note that cipherext list c now has a length of 18.
The feedback register f is now loaded with a 128-bit integer 
that is composed of ciphertext bytes (nth 2 c) through (nth 17 c), 
what RFC-4880 calls "resynchronization" (step 7).

The 128-bit value in f in encrypted using aes-cipher-128
and the 128-bit product is written into e (step 8).
This 128-bit product is xored with a 128-bit integer 
composed of the first 16 plaintext bytes in p,
and the 128-bit sum is split into 16 bytes and appended to ciphertext list c (step 9).
The feedback register f is loaded with the same 128-bit sum (step 10).

The 128-bit value in f in encrypted using aes-cipher-128
and the 128-bit product is written into e (step 11).
This 128-bit product is xored with a 128-bit integer 
composed of the next 16 plaintext bytes in p,
and the 128-bit sum is split into 16 bytes and appended to ciphertext list c.
The feedback register f is loaded with the same 128-bit sum (step 12).
Steps 11 and 12 are repeated until there are no more bytes in plaintext list p.


\section*{Simplification}

The above algorithm is for a CFB encoder.
A decoder algorithm is not provided by RFC-4880.
Therefore it appears we must derive the decoder algorithm.
To do that, we need to simplify things
and figure out what is really going on.

And what's really going on is this:
With the exception of the first 18 bytes,
to get the current block of ciphertext ($c_{i}$)
we encrypt the previous block of ciphertext ($c_{i-1}$)
and xor it with the current block of plaintext ($p_{i}$).
\begin{displaymath}
  c_{i} = k(c_{i-1}) + p_{i}
\end{displaymath}

Note that the encryption function is "k" and the xor operator is "+".
Note also that we are able to specify the CFB algorithm without a feedback register.
We simply use subscripts.

We can reverse this process and recover the plaintext.
To get the current block of plaintext ($p_{i}$)
we encrypt the previous block of ciphertext ($c_{i-1}$)
and xor it with the current block of ciphertext ($c_{i}$).
\begin{displaymath}
  p_{i} = k(c_{i-1}) + c_{i}
\end{displaymath}

The proof is simple because of the xor identity $a + a = 0$.
We are given
\begin{displaymath}
  c_{i} = k(c_{i-1}) + p_{i}
\end{displaymath}

Now we simply "add" $k(c_{i-1})$ to both sides.
\begin{displaymath}
  c_{i} + k(c_{i-1}) = k(c_{i-1}) + p_{i} + k(c_{i-1})
\end{displaymath}

The xor operation is commutative, so on the right side, 
the two $k(c_{i-1})$ terms cancel because $a + a = 0$.
Rearranging the remaining terms we have
\begin{displaymath}
  p_{i} = k(c_{i-1}) + c_{i}
\end{displaymath}
Q.E.D.


What remains is to describe the first 18 bytes of ciphertext.
To compute the first 16-byte ciphertext block (which we'll call $c_{-2}$) 
we encrypt 0 and xor it with the 16 random bytes r.
\begin{displaymath}
  c_{-2} = k(0) + r
\end{displaymath}

We recover $r$ with $c_{-2}$.
The proof is the same as above, where we "add" $k(0)$ to both sides.
\begin{displaymath}
  r = k(0) + c_{-2}
\end{displaymath}

To compute the two "quick check" bytes (which we'll call $c_{q}$) 
we encrypt the previous output block ($c_{-2}$) and
strip off the most-significant 16 bits ($msw$) of the product, 
and then xor that with the last two random bytes in r ($r_{q}$).
\begin{displaymath}
  c_{q} = msw(k(c_{-2})) + r_{q}
\end{displaymath}

Again, we can recover $r_{q}$ the same way.
\begin{displaymath}
  r_{q} = msw(k(c_{-2})) + c_{q}
\end{displaymath}

To compute the next block of ciphertext ($c_{0}$) 
we encrypt the previous ciphertext block ($c_{-1}$)
and xor it with the first plaintext block ($p_{0}$).
The previous ciphertext block $c_{-1}$ is the list (subseq c 2 17).
\begin{displaymath}
  c_{0} = k(c_{-1}) + p_{0}
\end{displaymath}

To compute the next block of ciphertext ($c_{1}$) 
we encrypt the previous ciphertext block ($c_{0}$) 
and xor it with the next plaintext block ($p_{1}$).
The previous ciphertext block $c_{-1}$ is the list (subseq c 2 17).
\begin{displaymath}
  c_{1} = k(c_{0}) + p_{1}
\end{displaymath}


\section*{Demonstration}
We wish to demonstrate the basic encoder and decoder operations.
We first encode one block of plaintext x into two blocks of ciphertext f and y.
Next we decode those two blocks back into the one block of plaintext z.
Finally we compare the original plaintext x with the decoded plaintext z.
\begin{verbatim}

Make a 128-bit key k.
(setf k (random (expt 2 128)))

Make the 44-element key-schedule array w.
(setf w (aes-expand-128 k))

Make the original plaintext block x.
(setf x (random (expt 2 128)))

Make the 1st ciphertext block f.
(setf f (random (expt 2 128)))

Make the 2nd ciphertext block y.
(setf y (logxor (aes-cipher-128 w f) x))

Make the decoded plaintext z.
(setf z (logxor (aes-cipher-128 w f) y))

See if they're the same.
(= x z)

Make the 2nd ciphertext block the 1st one.
(setf f y)

Make another plaintext block x.
(setf x (random (expt 2 128)))

Make another ciphertext block y.
(setf y (logxor (aes-cipher-128 w f) x))

Make another decoded plaintext z.
(setf z (logxor (aes-cipher-128 w f) y))

See if they're the same.
(= x z)

\end{verbatim}

