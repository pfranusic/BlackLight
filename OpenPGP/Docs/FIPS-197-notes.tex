%%%% FIPS-197-notes.tex
%%%% Copyright 2012 Peter Franusic
%%%% All rights reserved.
%%%%

\documentclass{article}
%\pagestyle{empty}

%%%% Various environments
\usepackage{verbatim}
\usepackage{graphicx}
\usepackage{latexsym}
\usepackage{amssymb}

%%%% Easy-vision mode
\usepackage[usenames]{color}
%\pagecolor{black}
%\color{green}

%%%% math-mode commands
\newcommand{\GF}{\mathrm{GF}(2^8)}
\newcommand{\gfadd}{\oplus}
\newcommand{\gfmul}{\bullet}

%%%% PDF metadata
\pdfinfo
{ /Title (FIPS-197 notes)
  /Author (Peter Franusic)
  /Subject (Advanced Encryption Standard)
  /Keywords (AES, cryptography, finite field)
}

%%%% European-style paragraphs
%%%% IMPORTANT: \begin{document} must follow for this to work.
\setlength{\parindent}{0pt} 
\setlength{\parskip}{1.3ex} 

%%%% Title block
\title{\textbf{\huge{FIPS-197 notes}}}
\author{Peter Franu\v si\'c
  \footnote{
    Copyright 2012 Peter Franu\v si\'c.
    All rights reserved.
    Email: \texttt{pete@sargo.com}}}
\date{}

\begin{document}
%%%%%%%%%%%%%%%%%%%%%%%%%%%%%%%%%%%%%%%%%%%%%%%%%%%%%%%%%%%%%%%%%%%%%%%%%%%%%%%%
\maketitle

FIPS-197 is the Federal Information Processing Standard Publication 197 (FIPS PUB 197).
It is a publication of the National Institute of Science and Technology (NIST).
FIPS-197 specifies the Advanced Encryption Standard (AES).
New readers of FIPS-197 may find themselves faced with a steep learning curve.
These notes address some of the more difficult topics.
In order to provide synchronization, the section headings are identical to 
those in FIPS-197.


%%%%%%%%%%%%%%%%%%%%%%%%%%%%%%%%%%%%%%%%%%%%%%%%%%%%%%%%%%%%%%%%%%%%%%%%%%%%%%%%
\section*{4. Mathematical Preliminaries}

\begin{quote}
  All bytes in the AES algorithm are interpreted as finite field elements\ldots
  Finite field elements can be added and multiplied, but these operations are 
  different from those used for numbers.
\end{quote}

A \emph{finite field} is a set with a finite number of elements and two operations.
AES uses the finite field $\GF$ where the GF stands for ``Galois Field'' (pronouced ``gal WAH'')
and the $2^8$ is the number of elements in the set.

$\GF$ has a set $\mathrm{P}$ that consists of 256 polynomials.
There are two operations in $\GF$.
The first is the ``add'' operation (denoted here by $\gfadd$).
The second is the ``multiply'' operation (denoted here by $\gfmul$).
\[ \GF = (\mathrm{P},\gfadd,\gfmul) \]

A few words about symbols\ldots
The ``add'' operation is denoted above by the symbol $\gfadd$.
FIPS-197 uses this symbol in some equations,
but in most equations it uses the $+$ symbol.
Both symbols signify the same operation: 
the ``addition'' of polynomial elements in $\GF$.
In these notes, we'll only use the $+$ symbol.

\newpage
The ``multiply'' operation is denoted above by the symbol $\gfmul$.
FIPS-197 uses this symbol in some equations.
In other equations it simply omits the symbol and the operation is implied.
In both cases the operation means ``multiplication'' of polynomial elements in $\GF$.
In these notes, we'll use the implied form in most equations.

We also note that the symbol $\otimes$ is introduced is section 2.2
as the symbol for modulo $x^4 + 1$ multiplication of polynomials with coefficients in $\GF$.
The symbol is used later in sections 4.3, 5.1.3 and 5.3.3.
The $\otimes$ symbol should not be confused with the $\gfmul$ symbol.

It is important to understand that the elements in $\GF$ are \emph{not} integers.
Rather, each element is a \emph{polynomial} with binary coefficients.
For example, the polynomial $x^5 + x^2$ is an element in $\GF$.
Any $\GF$ polynomial may be represented using an \emph{integer vector} notation
which consists of two hexadecimal digits within curly braces.

Here is a partial list of the $\GF$ polynomials and the corresponding integer vectors.
Note that the ``multiply'' operation is implied in each term.
Also note that the eight binary coefficients in each polynomial
are represented as two hexadecimal digits within curly braces.
\begin{eqnarray*}
  0 x^7 + 0 x^6 + 0 x^5 + 0 x^4 + 0 x^3 + 0 x^2 + 0 x^1 + 0 x^0  &=&  \{\mathtt{00}\} \\
  0 x^7 + 0 x^6 + 0 x^5 + 0 x^4 + 0 x^3 + 0 x^2 + 0 x^1 + 1 x^0  &=&  \{\mathtt{01}\} \\
  0 x^7 + 0 x^6 + 0 x^5 + 0 x^4 + 0 x^3 + 0 x^2 + 1 x^1 + 0 x^0  &=&  \{\mathtt{02}\} \\
  0 x^7 + 0 x^6 + 0 x^5 + 0 x^4 + 0 x^3 + 0 x^2 + 1 x^1 + 1 x^0  &=&  \{\mathtt{03}\} \\
  &\vdots& \\
  1 x^7 + 1 x^6 + 1 x^5 + 1 x^4 + 1 x^3 + 1 x^2 + 0 x^1 + 1 x^0  &=&  \{\mathtt{FD}\} \\
  1 x^7 + 1 x^6 + 1 x^5 + 1 x^4 + 1 x^3 + 1 x^2 + 1 x^1 + 0 x^0  &=&  \{\mathtt{FE}\} \\
  1 x^7 + 1 x^6 + 1 x^5 + 1 x^4 + 1 x^3 + 1 x^2 + 1 x^1 + 1 x^0  &=&  \{\mathtt{FF}\}
\end{eqnarray*}

We usually write $\GF$ polynomials using a short-hand notation.
Any term with its coefficient equal to 0 is not written,
any term with its coefficient equal to 1 is written without the coefficient,
any term $x^1$ is written simply as $x$, and any term $x^0$ is written as 1.
For example
\begin{eqnarray*}
  x  &=&  \{\mathtt{02}\} \\
  x^2 + 1  &=&  \{\mathtt{05}\} \\
  x^7 + x^3 + x + 1  &=&  \{\mathtt{8B}\}
\end{eqnarray*}


\newpage
%%%%%%%%%%%%%%%%%%%%%%%%%%%%%%%%%%%%%%%%%%%%%%%%%%%%%%%%%%%%%%%%%%%%%%%%%%%%%%%%
\subsection*{4.1 Addition}

\begin{quote}
  The addition of two elements in a finite field is achieved by ``adding''
  the coefficients for the corresponding powers in the polynomials for the 
  two elements.  The addition is performed with the XOR operation\ldots
\end{quote}

The ``add'' operation in $\GF$ is \emph{not} integer addition.
It is polynomial addition.
The following example in FIPS-197 shows the addition of two polynomials.
\begin{eqnarray*}
 (x^6+x^4+x^2+x+1) + (x^7+x+1) &=& x^7 + x^6 + x^4 + x^2
\end{eqnarray*}

So what happened to the $x+1$ in each polynomial?
Why didn't they show up in the sum as $2x+2$?
The answer is that we're not doing integer addition.
The $+$ operator denotes exclusive-or addition.
Here's the exclusive-or truth table:
\begin{eqnarray*}
  0 + 0 &=& 0 \\
  0 + 1 &=& 1 \\
  1 + 0 &=& 1 \\
  1 + 1 &=& 0
\end{eqnarray*}

In the example, the two $x$ terms cancel out because
the exclusive-or of their coefficients is 0.
The distributive property holds for $\GF$.
\begin{eqnarray*}
  x + x &=& 1 x + 1 x \\
  &=& (1+1) x \\
  &=& 0 x \\
  &=& 0
\end{eqnarray*}


%%%%%%%%%%%%%%%%%%%%%%%%%%%%%%%%%%%%%%%%%%%%%%%%%%%%%%%%%%%%%%%%%%%%%%%%%%%%%%%%
\subsection*{4.2 Multiplication}

\begin{quote}
  In the polynomial representation, multiplication in $\GF$ 
  (denoted by $\gfmul$) corresponds with the multiplication 
  of polynomials modulo an \textbf{irreducible polynomial}
  of degree 8.
\end{quote}

The ``multiply'' operation in $\GF$ is \emph{not} integer multiplication.
It is modular multiplication of polynomials.
The modulus is the polynomial $m(x)$.
It has an $x^8$ term, which makes it a polynomial of degree 8.
\begin{eqnarray*}
 m(x) &=& x^8 + x^4 + x^3 + x + 1
\end{eqnarray*}

The polynomial $m(x)$ is \emph{irreducible} because it cannot be factored.
It's similar to a prime integer.
Except for 1 and $m(x)$, there are no two polynomials $p(x)$ and $q(x)$ such that
\begin{eqnarray*}
p(x) \gfmul q(x) &=& m(x)
\end{eqnarray*}

The following example in FIPS-197 shows the multiplication of two polynomials.
Note that the distribution property holds for $\GF$.
\begin{eqnarray*}
  \{\mathtt{57}\} \gfmul \{\mathtt{83}\}
  &=& (x^6 + x^4 + x^2 + x + 1) \gfmul (x^7 + x + 1) \\
  &=& (x^6 + x^4 + x^2 + x + 1) \gfmul x^7 + \\
  & & (x^6 + x^4 + x^2 + x + 1) \gfmul x + \\
  & & (x^6 + x^4 + x^2 + x + 1) \gfmul 1 \\
  &=& x^{13} + x^{11} + x^9 + x^8 + x^7 + \\
  & & x^7 + x^5 + x^3 + x^2 + x + \\
  & & x^6 + x^4 + x^2 + x + 1 \\
  &=& x^{13} + x^{11} + x^9 + x^8 + x^6 + x^5 + x^4 + x^3 + 1
\end{eqnarray*}

We have a problem. This is a degree 13 polynomial.
$\GF$ polynomials are no larger that degree 7.
So what went wrong?
The answer is that ``multiplication'' in $\GF$ includes reduction by $m(x)$.
We neglected to reduce on each multiply above.
But fortunately, because of the distribution property,
we can do one humongous reduction at the end,
which is what the example in FIPS-197 does.

We can compute the reduction using long division.
The dividend is the degree 13 polynomial from above.
The divisor is our irreducible degree 8 polynomial $m(x)$.
We do the long division, chuck the quotient polynomial $x^5 + x^3$
and keep the remainder polynomial $x^7 + x^6 + 1$.
Remember that ``subtraction'' in $\GF$ is simply an exclusive-or,
just like ``addition.''

We show the details of the long division as two subtractions.
For the first subtraction, we multiply our degree 8 $m(x)$ by $x^5$
so that we have an $x^{13}$ term that will cancel out.
\begin{eqnarray*}
  m(x) \gfmul x^5 &=& (x^8 + x^4 + x^3 + x + 1) \gfmul x^5 \\
  &=& x^{13} + x^9 + x^8 + x^6 + x^5
\end{eqnarray*}

Now we ``subtract'' this from the degree 13 polynomial.
To ``subtract'' we simply do the exclusive-or operation $\gfadd$ in $\GF$.
(\emph{E.g.} $x^{13} - x^{13} = x^{13} \gfadd x^{13} = 0$).
We are left with a polynomial of degree 11.
\begin{eqnarray*}
  & & x^{13} + x^{11} + x^9 + x^8 + x^6 + x^5 + x^4 + x^3 + 1 \\
  &-& (x^{13} + x^9 + x^8 + x^6 + x^5) \\
  &=& x^{11} + x^4 + x^3 + 1
\end{eqnarray*}

For the second subtraction, we multiply $m(x)$ by $x^3$
so that we have an $x^{11}$ term that will cancel out.
\begin{eqnarray*}
  m(x) \gfmul x^3 &=& (x^8 + x^4 + x^3 + x + 1) \gfmul x^3 \\
  &=& x^{11} + x^7 + x^6 + x^4 + x^3
\end{eqnarray*}

Now we ``subtract'' this from the degree 11 polynomial.
We are left with a polynomial of degree 7, 
which is in $\GF$ and is the element $\{\mathtt{C1}\}$.
\begin{eqnarray*}
  & & x^{11} + x^4 + x^3 + 1 \\
  &-& (x^{11} + x^7 + x^6 + x^4 + x^3) \\
  &=& x^7 + x^6 + 1 \\
  &=& \{\mathtt{C1}\}
\end{eqnarray*}


%%%%%%%%%%%%%%%%%%%%%%%%%%%%%%%%%%%%%%%%%%%%%%%%%%%%%%%%%%%%%%%%%%%%%%%%%%%%%%%%
\subsection*{4.3 Polynomials with Coefficients in $\GF$}

We are given the two polynomials $a(x)$ and $b(x)$. Each has four terms.
Each term is the product of a coefficient (\emph{e.g.} $a_3$) 
and a power of $x$ (\emph{e.g.} $x^3$).
The coefficients of $a(x)$ and $b(x)$ are not integers. They are polynomials in $\GF$.
\begin{eqnarray*}
  a(x) &=& a_3 x^3 + a_2 x^2 + a_1 x + a_0 \\
  b(x) &=& b_3 x^3 + b_2 x^2 + b_1 x + b_0
\end{eqnarray*}

We wish to compute the modulo $x^4 + 1$ product $a(x) \otimes b(x)$.
The example in FIPS-197 does this in two steps.
The first step is to do the multiplications without reductions.
We save the reduction for the second step.
\begin{eqnarray*}
  a(x) \otimes b(x)
  &=& (a_3 x^3 + a_2 x^2 + a_1 x + a_0) \otimes \\
  & & (b_3 x^3 + b_2 x^2 + b_1 x + b_0) \\
  &=& (a_3 x^3 + a_2 x^2 + a_1 x + a_0) b_3 x^3 + \\
  & & (a_3 x^3 + a_2 x^2 + a_1 x + a_0) b_2 x^2 + \\
  & & (a_3 x^3 + a_2 x^2 + a_1 x + a_0) b_1 x + \\
  & & (a_3 x^3 + a_2 x^2 + a_1 x + a_0) b_0 \\
  &=& a_3 b_3 x^6 + a_2 b_3 x^5 + a_1 b_3 x^4 + a_0 b_3 x^3 + \\
  & & a_3 b_2 x^5 + a_2 b_2 x^4 + a_1 b_2 x^3 + a_0 b_2 x^2 + \\
  & & a_3 b_1 x^4 + a_2 b_1 x^3 + a_1 b_1 x^2 + a_0 b_1 x + \\
  & & a_3 b_0 x^3 + a_2 b_0 x^2 + a_1 b_0 x + a_0 b_0 \\
  &=& (a_3 b_3) x^6 + (a_3 b_2 + a_2 b_3) x^5 + \\
  & & (a_3 b_1 + a_2 b_2 + a_1 b_3) x^4 + \\
  & & (a_3 b_0 + a_2 b_1 + a_1 b_2 + a_0 b_3) x^3 + \\
  & & (a_2 b_0 + a_1 b_1 + a_0 b_2) x^2 + \\
  & & (a_1 b_0 + a_0 b_1) x + a_0 b_0
\end{eqnarray*}

\newpage
We replace each of the seven coefficient expressions with a $c_i$
so that we have a compact set of equations.
The result is the same as equation (4.9) in FIPS-197.
\begin{eqnarray*}
  a(x) \otimes b(x) &=& c(x) \\
  c(x) &=& c_6 x^6 + c_5 x^5 + c_4 x^4 + c_3 x^3 + c_2 x^2 + c_2 x + c_0 \\
  c_6 &=& a_3 b_3 \\
  c_5 &=& a_3 b_2 + a_2 b_3 \\
  c_4 &=& a_3 b_1 + a_2 b_2 + a_1 b_3 \\
  c_3 &=& a_3 b_0 + a_2 b_1 + a_1 b_2 + a_0 b_3 \\
  c_2 &=& a_2 b_0 + a_1 b_1 + a_0 b_2 \\
  c_1 &=& a_1 b_0 + a_0 b_1 \\
  c_0 &=& a_0 b_0
\end{eqnarray*}

Now for step two.
We have $c(x)$ which is a degree 6 polynomial.
It must be reduced to a degree 3 polynomial.
Again, we can compute the reduction using long division
where the dividend is $c(x)$ and the divisor is $x^4 + 1$.
The reduction requires three subtractions.

For the first subtraction, we multiply $x^4 + 1$ by $c_6 x^2$
so that we have an $x^6$ term that will cancel out.
\begin{eqnarray*}
  (x^4 + 1) c_6 x^2 &=& c_6 x^6 + c_6 x^2
\end{eqnarray*}

Now we ``subtract'' this from the $c(x)$ polynomial.
Since $(c_6 + c_6)=0$, we are left with a polynomial of degree 5.
\begin{eqnarray*}
  & & (c_6 x^6 + c_5 x^5 + c_4 x^4 + c_3 x^3 + c_2 x^2 + c_1 x + c_0) \\
  &-& (c_6 x^6 + c_6 x^2) \\
  &=& (c_6 + c_6) x^6 + c_5 x^5 + c_4 x^4 + c_3 x^3 + (c_2 + c_6) x^2 + c_1 x + c_0 \\
  &=& c_5 x^5 + c_4 x^4 + c_3 x^3 + (c_2 + c_6) x^2 + c_1 x + c_0
\end{eqnarray*}

For the second subtraction, we multiply $x^4 + 1$ by $c_5 x$
so that we have an $x^5$ term that will cancel out.
\begin{eqnarray*}
  (x^4 + 1) c_5 x &=& c_5 x^5 + c_5 x
\end{eqnarray*}

This gets ``subtracted'' from the degree 5 remainder above.
Since $(c_5 + c_5)=0$, we are left with a polynomial of degree 4.
\begin{eqnarray*}
  & & (c_5 x^5 + c_4 x^4 + c_3 x^3 + (c_2 + c_6) x^2 + c_1 x + c_0) \\
  &-& (c_5 x^5 + c_5 x) \\
  &=& (c_5 + c_5) x^5 + c_4 x^4 + c_3 x^3 + (c_2 + c_6) x^2 + (c_1 + c_5) x + c_0 \\
  &=& c_4 x^4 + c_3 x^3 + (c_2 + c_6) x^2 + (c_1 + c_5) x + c_0
\end{eqnarray*}

For the last subtraction, we multiply $x^4 + 1$ by $c_4$
so that we have an $x^4$ term that will cancel out.
\begin{eqnarray*}
  (x^4 + 1) c_4 &=& c_4 x^4 + c_4
\end{eqnarray*}

This gets ``subtracted'' from the degree 4 remainder above.
Since $(c_4 + c_4)=0$, we are left with a polynomial of degree 3,
which is what we want.
\begin{eqnarray*}
  & & (c_4 x^4 + c_3 x^3 + (c_2 + c_6) x^2 + (c_1 + c_5) x + c_0) \\
  &-& (c_4 x^4 + c_4) \\
  &=& (c_4 + c_4) x^4 + c_3 x^3 + (c_2 + c_6) x^2 + (c_1 + c_5) x + (c_0 + c_4) \\
  &=& c_3 x^3 + (c_2 + c_6) x^2 + (c_1 + c_5) x + (c_0 + c_4)
\end{eqnarray*}

We replace the four coefficient expressions with a $d_i$
where each $d_i$ is expressed using the coefficients from $a(x)$ and $b(x)$.
The result is the same as equation (4.12) in FIPS-197.
\begin{eqnarray*}
  a(x) \otimes b(x) &=& d(x) \\
  d(x) &=& d_3 x^3 + d_2 x^2 + d_2 x + d_0 \\
  d_3 &=& c_3 = a_3 b_0 + a_2 b_1 + a_1 b_2 + a_0 b_3 \\
  d_2 &=& (c_2 + c_6) = a_2 b_0 + a_1 b_1 + a_0 b_2 + a_3 b_3 \\
  d_1 &=& (c_1 + c_5) = a_1 b_0 + a_0 b_1 + a_3 b_2 + a_2 b_3 \\
  d_0 &=& (c_0 + c_4) = a_0 b_0 + a_3 b_1 + a_2 b_2 + a_1 b_3
\end{eqnarray*}

Equations (4.14) and (4.15) specify two polynomials $a(x)$ and $a^{-1}(x)$.
Polynomial $a(x)$ is used in the \texttt{MixColumns} transformation (section 5.1.3).
Polynomial $a^{-1}(x)$ is used in the \texttt{InvMixColumns} transformation (section 5.3.3).
\begin{eqnarray*}
  a(x) &=& \{\mathtt{03}\}x^3 + \{\mathtt{01}\}x^2 + \{\mathtt{01}\}x + \{\mathtt{02}\} \\
  a^{-1}(x) &=& \{\mathtt{0B}\}x^3 + \{\mathtt{0D}\}x^2 + \{\mathtt{09}\}x + \{\mathtt{0E}\}
\end{eqnarray*}

What makes \texttt{MixColumns} and \texttt{InvMixColumns} work together
is that $a^{-1}(x)$ is the multiplicative inverse of $a(x)$.
In other words, the modulo $x^4+1$ product of $a(x)$ and $a^{-1}(x)$ 
is the multiplicative identity polynomial $\{\mathtt{01}\}$.
\begin{eqnarray*}
  a(x) \otimes a^{-1}(x) &=& \{\mathtt{01}\}
\end{eqnarray*}

\newpage
The coefficients of $d(x)$ give us an easy way to verify this.
First we write the product in terms of $d(x)$.
Then we use the formulas in equation (4.12) to compute 
the four coefficients of $d(x)$.
\begin{eqnarray*}
  a(x) \otimes a^{-1}(x) &=& d(x) \\
  &=& d_3 x^3 + d_2 x^2 + d_1 x + d_0
\end{eqnarray*}
\begin{eqnarray*}
  d_3 &=& a_3 b_0 + a_2 b_1 + a_1 b_2 + a_0 b_3 \\
      &=& \{\mathtt{03}\} \{\mathtt{0E}\} + 
          \{\mathtt{01}\} \{\mathtt{09}\} + 
	  \{\mathtt{01}\} \{\mathtt{0D}\} + 
	  \{\mathtt{02}\} \{\mathtt{0B}\} \\
      &=& \{\mathtt{03}\} + \{\mathtt{09}\} + \{\mathtt{0D}\} + \{\mathtt{07}\} \\
      &=& \{\mathtt{00}\}
\end{eqnarray*}
\begin{eqnarray*}
  d_2 &=& a_2 b_0 + a_1 b_1 + a_0 b_2 + a_3 b_3 \\
      &=& \{\mathtt{01}\} \{\mathtt{0E}\} + 
          \{\mathtt{01}\} \{\mathtt{09}\} + 
	  \{\mathtt{02}\} \{\mathtt{0D}\} + 
	  \{\mathtt{03}\} \{\mathtt{0B}\} \\
      &=& \{\mathtt{0E}\} + \{\mathtt{09}\} + \{\mathtt{0B}\} + \{\mathtt{0C}\} \\
      &=& \{\mathtt{00}\}
\end{eqnarray*}
\begin{eqnarray*}
  d_1 &=& a_1 b_0 + a_0 b_1 + a_3 b_2 + a_2 b_3 \\
      &=& \{\mathtt{01}\} \{\mathtt{0E}\} + 
          \{\mathtt{02}\} \{\mathtt{09}\} + 
	  \{\mathtt{03}\} \{\mathtt{0D}\} + 
	  \{\mathtt{01}\} \{\mathtt{0B}\} \\
      &=& \{\mathtt{0E}\} + \{\mathtt{03}\} + \{\mathtt{06}\} + \{\mathtt{0B}\} \\
      &=& \{\mathtt{00}\}
\end{eqnarray*}
\begin{eqnarray*}
  d_0 &=& a_0 b_0 + a_3 b_1 + a_2 b_2 + a_1 b_3 \\
      &=& \{\mathtt{02}\} \{\mathtt{0E}\} + 
          \{\mathtt{03}\} \{\mathtt{09}\} + 
	  \{\mathtt{01}\} \{\mathtt{0D}\} + 
	  \{\mathtt{01}\} \{\mathtt{0B}\} \\
      &=& \{\mathtt{0D}\} + \{\mathtt{0A}\} + \{\mathtt{0D}\} + \{\mathtt{0B}\} \\
      &=& \{\mathtt{01}\}
\end{eqnarray*}

We assemble the four coefficients back into the equation for $d(x)$.
The first three coefficients are $\{\mathtt{00}\}$ so those terms cancel.
The coefficient for the $x^0$ term is $\{\mathtt{01}\}$. QED.
\begin{eqnarray*}
  d(x) &=& \{\mathtt{00}\}x^3 + \{\mathtt{00}\}x^2 + \{\mathtt{00}\}x + \{\mathtt{01}\} \\
       &=& \{\mathtt{01}\}
\end{eqnarray*}

%%%%%%%%%%%%%%%%%%%%%%%%%%%%%%%%%%%%%%%%%%%%%%%%%%%%%%%%%%%%%%%%%%%%%%%%%%%%%%%%
\subsection*{5.1.1 \texttt{SubBytes()} Transformation}

\begin{quote}
  The \texttt{SubBytes()} transformation is a non-linear byte substitution 
  that operates independently on each byte of the State using a substitution 
  table (S-box). This S-box, which is invertible, is constructed by composing 
  two transformations.
\end{quote}

The S-box is a non-linear substitution table 
used in the \texttt{SubBytes} and \texttt{SubWord} functions
to perform one-for-one substitution of byte values.
The S-box consists of 256 bytes that are precomputed.

The S-box for AES was not concocted in some dark corner
of the National Security Agency.
Section 5.1.1 in FIPS-197 specifies a two-step algorithm 
to compute each byte in the S-box.
\begin{enumerate}
\item
  Compute the GF(256) \emph{multiplicative inverse}
  using the irreducible polynomial m(x) given in 
  equation (4.1) in FIPS-197.
\item
  Compute the eight GF(2) \emph{affine transformations}
  using the formula given in equation (5.1) in FIPS-197.
\end{enumerate}

The following table shows the multiplicative inverse
for each polynomial element of GF(256) except $\{\mathtt{00}\}$.
The irreducible polynomial $m(x) = \{\mathtt{011B}\}$ is the modulus.
The table is organized into 16 rows and 16 columns.
Given a polynomial represented by the two hexadecimal digits $\{\mathtt{RC}\}$,
the multiplicative inverse is at the intersection of row R and column C.
\begin{verbatim}
     01 8D F6 CB 52 7B D1 E8 4F 29 C0 B0 E1 E5 C7
  74 B4 AA 4B 99 2B 60 5F 58 3F FD CC FF 40 EE B2
  3A 6E 5A F1 55 4D A8 C9 C1 0A 98 15 30 44 A2 C2
  2C 45 92 6C F3 39 66 42 F2 35 20 6F 77 BB 59 19
  1D FE 37 67 2D 31 F5 69 A7 64 AB 13 54 25 E9 09
  ED 5C 05 CA 4C 24 87 BF 18 3E 22 F0 51 EC 61 17
  16 5E AF D3 49 A6 36 43 F4 47 91 DF 33 93 21 3B
  79 B7 97 85 10 B5 BA 3C B6 70 D0 06 A1 FA 81 82
  83 7E 7F 80 96 73 BE 56 9B 9E 95 D9 F7 02 B9 A4
  DE 6A 32 6D D8 8A 84 72 2A 14 9F 88 F9 DC 89 9A
  FB 7C 2E C3 8F B8 65 48 26 C8 12 4A CE E7 D2 62
  0C E0 1F EF 11 75 78 71 A5 8E 76 3D BD BC 86 57
  0B 28 2F A3 DA D4 E4 0F A9 27 53 04 1B FC AC E6
  7A 07 AE 63 C5 DB E2 EA 94 8B C4 D5 9D F8 90 6B
  B1 0D D6 EB C6 0E CF AD 08 4E D7 E3 5D 50 1E B3
  5B 23 38 34 68 46 03 8C DD 9C 7D A0 CD 1A 41 1C
\end{verbatim}

For example, the multiplicative inverse of $\{\mathtt{53}\}$
may be found at the intersection of row 5 and column 3.
Therefore the multiplicative inverse of $\{\mathtt{53}\}$ is $\{\mathtt{CA}\}$.
I.e., $\{\mathtt{53}\} \{\mathtt{CA}\} = \{\mathtt{01}\}$.
Note that element $\{\mathtt{00}\}$ does not have an inverse (it's blank),
and no element in the table has the value $\{\mathtt{00}\}$.
Note also that the inverse of $\{\mathtt{01}\}$ is $\{\mathtt{01}\}$.

\newpage
Equation (5.1) can be expanded into the eight 
affine transformations shown here.
The plus sign ($+$) is the exclusive-or operator.
Note that the $c$ bits have been replaced by binary constants.
Note also that these eight transformations can be
computed in parallel by precomputing four additional 
bytes which provide the required rotations.
\begin{eqnarray*}
  b'_7 &=& b_7 + b_3 + b_4 + b_5 + b_6 + 0 \\
  b'_6 &=& b_6 + b_2 + b_3 + b_4 + b_5 + 1 \\
  b'_5 &=& b_5 + b_1 + b_2 + b_3 + b_4 + 1 \\
  b'_4 &=& b_4 + b_0 + b_1 + b_2 + b_3 + 0 \\
  b'_3 &=& b_3 + b_7 + b_0 + b_1 + b_2 + 0 \\
  b'_2 &=& b_2 + b_6 + b_7 + b_0 + b_1 + 0 \\
  b'_1 &=& b_1 + b_5 + b_6 + b_7 + b_0 + 1 \\
  b'_0 &=& b_0 + b_4 + b_5 + b_6 + b_7 + 1
\end{eqnarray*}

The following table is the S-box shown in FIPS-197 Figure 7.
Each polynomial in the table has been computed using the two-step algorithm.
The table is organized into 16 rows and 16 columns.
Given a polynomial represented by the two hexadecimal digits $\{\mathtt{RC}\}$,
the substitution polynomial is at the intersection of row R and column C.
\begin{verbatim}
  63 7C 77 7B F2 6B 6F C5 30 01 67 2B FE D7 AB 76
  CA 82 C9 7D FA 59 47 F0 AD D4 A2 AF 9C A4 72 C0
  B7 FD 93 26 36 3F F7 CC 34 A5 E5 F1 71 D8 31 15
  04 C7 23 C3 18 96 05 9A 07 12 80 E2 EB 27 B2 75
  09 83 2C 1A 1B 6E 5A A0 52 3B D6 B3 29 E3 2F 84
  53 D1 00 ED 20 FC B1 5B 6A CB BE 39 4A 4C 58 CF
  D0 EF AA FB 43 4D 33 85 45 F9 02 7F 50 3C 9F A8
  51 A3 40 8F 92 9D 38 F5 BC B6 DA 21 10 FF F3 D2
  CD 0C 13 EC 5F 97 44 17 C4 A7 7E 3D 64 5D 19 73
  60 81 4F DC 22 2A 90 88 46 EE B8 14 DE 5E 0B DB
  E0 32 3A 0A 49 06 24 5C C2 D3 AC 62 91 95 E4 79
  E7 C8 37 6D 8D D5 4E A9 6C 56 F4 EA 65 7A AE 08
  BA 78 25 2E 1C A6 B4 C6 E8 DD 74 1F 4B BD 8B 8A
  70 3E B5 66 48 03 F6 0E 61 35 57 B9 86 C1 1D 9E
  E1 F8 98 11 69 D9 8E 94 9B 1E 87 E9 CE 55 28 DF
  8C A1 89 0D BF E6 42 68 41 99 2D 0F B0 54 BB 16
\end{verbatim}

For example, given the polynomial $\{\mathtt{53}\}$, its substitution 
may be found at the intersection of row 5 and column 3.
Therefore the substitution of $\{\mathtt{53}\}$ is $\{\mathtt{ED}\}$.
Note from the example above that the multiplicative inverse 
of $\{\mathtt{53}\}$ is $\{\mathtt{CA}\}$.
And the eight affine transformations applied to 
$\{\mathtt{CA}\}$ produces $\{\mathtt{ED}\}$.


%%%%%%%%%%%%%%%%%%%%%%%%%%%%%%%%%%%%%%%%%%%%%%%%%%%%%%%%%%%%%%%%%%%%%%%%%%%%%%%%
\subsection*{5.3.2 \texttt{InvSubBytes()} Transformation}

\begin{quote}
  \texttt{InvSubBytes()} is the inverse of the byte substitution transformation, 
  in which the inverse S-box is applied to each byte of the State.
  This is obtained by applying the inverse of the affine transformation
  followed by taking the multiplicative inverse in $\GF$.
\end{quote}

The inverse S-box is also non-linear substitution table.
It is used in the \texttt{InvSubBytes} function
to perform one-for-one substitution of byte values.
The inverse S-box consists of 256 bytes that are precomputed
using a two-step algorithm.
\begin{enumerate}
\item
  Compute the eight \emph{inverse affine transformations}
  using the formula given below.
\item
  Compute the GF(256) \emph{multiplicative inverse}
  using the polynomial m(x).
\end{enumerate}

The inverse affine transformation can be derived
from the original eight affine transformations.
The first step is to rewrite each equation.
For example, in the first equation, we ``add'' $b'_7 + b_7$ to both sides.
The $b'_7$ on the left drops out, and the $b_7$ on the right drops out.
\begin{eqnarray*}
  b_7 &=& b'_7 + b_3 + b_4 + b_5 + b_6 + 0 \\
  b_6 &=& b'_6 + b_2 + b_3 + b_4 + b_5 + 1 \\
  b_5 &=& b'_5 + b_1 + b_2 + b_3 + b_4 + 1 \\
  b_4 &=& b'_4 + b_0 + b_1 + b_2 + b_3 + 0 \\
  b_3 &=& b'_3 + b_7 + b_0 + b_1 + b_2 + 0 \\
  b_2 &=& b'_2 + b_6 + b_7 + b_0 + b_1 + 0 \\
  b_1 &=& b'_1 + b_5 + b_6 + b_7 + b_0 + 1 \\
  b_0 &=& b'_0 + b_4 + b_5 + b_6 + b_7 + 1
\end{eqnarray*}

Each equation now has an ``old'' bit value on the left
and a ``new'' bit value on the right, along with some old ones.
The idea is to replace each of the old bit values on the right
with the appropriate equations, so that all of the old bit values 
cancel each other out and we are left with only new bit values on the right.
Here's the results:
\begin{eqnarray*}
  b_7 &=& b'_6 + b'_4 + b'_1 + 0 \\
  b_6 &=& b'_5 + b'_3 + b'_0 + 0 \\
  b_5 &=& b'_4 + b'_2 + b'_7 + 0 \\
  b_4 &=& b'_3 + b'_1 + b'_6 + 0 \\
  b_3 &=& b'_2 + b'_0 + b'_5 + 0 \\
  b_2 &=& b'_1 + b'_7 + b'_4 + 1 \\
  b_1 &=& b'_0 + b'_6 + b'_3 + 0 \\
  b_0 &=& b'_7 + b'_5 + b'_2 + 1
\end{eqnarray*}

These eight equations can be condensed into one equation
that is similar to equation (5.1) in FIPS-197.
\begin{eqnarray*}
  b_i &=& b'_{(i+7) \bmod 8} \gfadd
          b'_{(i+5) \bmod 8} \gfadd
	  b'_{(i+2) \bmod 8} \gfadd
	  c'_i
\end{eqnarray*}
for $0 \le i < 8$, where $b'_i$ is the $i^{\mathrm{th}}$ bit of the byte,
and $c'_i$ is the $i^{\mathrm{th}}$ bit of a byte $c'$ with the value $\{\mathtt{05}\}$.
The following table is the inverted S-box shown in FIPS-197 Figure 14.
\begin{verbatim}
  52 09 6A D5 30 36 A5 38 BF 40 A3 9E 81 F3 D7 FB
  7C E3 39 82 9B 2F FF 87 34 8E 43 44 C4 DE E9 CB
  54 7B 94 32 A6 C2 23 3D EE 4C 95 0B 42 FA C3 4E
  08 2E A1 66 28 D9 24 B2 76 5B A2 49 6D 8B D1 25
  72 F8 F6 64 86 68 98 16 D4 A4 5C CC 5D 65 B6 92
  6C 70 48 50 FD ED B9 DA 5E 15 46 57 A7 8D 9D 84
  90 D8 AB 00 8C BC D3 0A F7 E4 58 05 B8 B3 45 06
  D0 2C 1E 8F CA 3F 0F 02 C1 AF BD 03 01 13 8A 6B
  3A 91 11 41 4F 67 DC EA 97 F2 CF CE F0 B4 E6 73
  96 AC 74 22 E7 AD 35 85 E2 F9 37 E8 1C 75 DF 6E
  47 F1 1A 71 1D 29 C5 89 6F B7 62 0E AA 18 BE 1B
  FC 56 3E 4B C6 D2 79 20 9A DB C0 FE 78 CD 5A F4
  1F DD A8 33 88 07 C7 31 B1 12 10 59 27 80 EC 5F
  60 51 7F A9 19 B5 4A 0D 2D E5 7A 9F 93 C9 9C EF
  A0 E0 3B 4D AE 2A F5 B0 C8 EB BB 3C 83 53 99 61
  17 2B 04 7E BA 77 D6 26 E1 69 14 63 55 21 0C 7D
\end{verbatim}


%%%%%%%%%%%%%%%%%%%%%%%%%%%%%%%%%%%%%%%%%%%%%%%%%%%%%%%%%%%%%%%%%%%%%%%%%%%%%%%%
\end{document}

